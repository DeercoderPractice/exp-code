% -*-coding: utf-8 -*-
% File: preface/cover.tex
%
% 修改自: PlutoThesis_UTF8_1.9.4.20100419.zip
%         http://code.google.com/p/plutothesis/
%
%
%% 封面和摘要


%% 封面内容

% 授权书中用到的论文题目, 无需手动断行
% 使用该命令的文件:
%    appendix/Authorization.tex

\newcommand{\chinesethesistitle}{哈尔滨工程大学本硕博学位论文~\LaTeX~模板~(\version~版)}



% 英文论文题目
% Tip: \uppercase 作用是将英文标题字母全部大写

\newcommand{\englishthesistitle}{\uppercase{\LaTeX~Dissertation Template of \\Harbin Engineering University~(Version \version)}}



% 封面底部的中英文日期

\newcommand{\chinesethesistime}{2010~年~6~月}
\newcommand{\englishthesistime}{June, 2010}



%% 中文封面论文标题, 学科学位, 学科, 院系, 姓名, 导师姓名, 日期

\ctitle{哈尔滨工程大学本硕博学位论文\\ \LaTeX~模板~(\version~版)}  % 论文标题, 可手动断行
\cdegree{\cxueke\cxuewei}  % \cxueke 在 setup/Definition.tex 中定义, \cxuewei 在 setup/type.tex 中定义
\csubject{计算机科学与技术}  % 学科
\caffil{计算机科学与技术学院}  % 院系
\cauthor{某某某}  % 或{某~~~~某}或超过三个字亦可
\csupervisor{某某某~~~~教授}  % 导师名字
%\cassosupervisor{某~~~~某~~~~教授}  % 副导师, 如无可以不列此项
%\ccosupervisor{某某某~~~~教授~}  % 联合培养导师, 如无可以不列此项
\cdate{\chinesethesistime}  % 中文日期



%% 英文封面论文标题, 学科学位, 学科, 院系, 姓名, 导师姓名, 日期

\etitle{\englishthesistitle}  % 英文标题
\edegree{\exuewei \ of \exueke}  % \exueke 在 setup/Definition.tex 中定义, \exuewei 在 setup/type.tex 中定义
\esubject{Computer\hfill Science\hfill and\hfill Technology}  % 英文二级学科名
\eaffil{College\hfill of\hfill Computer\hfill Science\newline and Technology}  % 英文院系名, 如需换行请用 \newline, 不要用\\
\eauthor{XXX}  % 英文作者姓名
\esupervisor{Professor XXX}  % 英文导师姓名
%\ecosupervisor{Professor YYY}
%\eassosupervisor{Professor ZZZ}
\edate{\englishthesistime}  % 英文日期



%% 图书分类号和密级

\natclassifiedindex{TP309}  % 中图分类号
\internatclassifiedindex{681.324}  % 国际图书分类号
\statesecrets{公开} % 或{秘密}



%% 摘要内容

%\iffalse
%\BiAppendixChapter{摘~~~~要}{}  % 使用 winedt 编辑时, 为了在文档结构图(toc)中显示摘要, 需增加此句
%\fi



% 中文摘要
\cabstract{
本模板是修改自哈尔滨工业大学 PlutoThesis 模板,并依照哈尔滨工程大学论文规范制作的\LaTeX{}学位论文模板。
我们的目标是形成一个符合哈工程论文规范,并且方便易用的本硕博学位论文模板。
写作本模板也旨在推广\LaTeX{}这一优秀的排版软件在哈工程的应用,为广大同学提供一个方便、
美观的论文模板,减少论文撰写方面的麻烦。
然而当前模板的制作远未完成,只有本科部分形成了雏形,硕博部分甚至尚未涉及,
因此也希望更多老师同学参与到模板的写作与维护中来,共同完成这个项目。

目前本模板已由 Yuliang(2010) 等人按照哈尔滨工程大学本科毕业设计(论文)手册,
完成了本科论文的大部分排版工作,后续的工作仍在继续中。
本模板的启动和制作过程得到了工大论文模板维护者 luckfox 的大力支持,在此表示感谢。

当然这个模板文件仅仅是一个开始,
希望有``牛人''能够综合这些设置形成真正的文档类形式(cls)的模板文件,造福以后的兄弟姐妹们。
不过补充一下,在目前需要多人参与维护的情况下,book 类的文档也具有一些自己的优势,
大家都很容易看懂代码,上手修改,也易于从这个模板开始学习\LaTeX{}的使用。二者各有特色吧。
总体上来说,当前这种类型的模板也还是很值得推荐使用的。
}

% 关键词
\ckeywords{\LaTeX{};论文模板;哈工程;(3 $\sim$ 6 个!)}



% 英文摘要
\eabstract{
This is a \LaTeX{} dissertation template of Harbin Engineering University, which is built according to the required format.
}

% 关键词
\ekeywords{\LaTeX{}; dissertation template; harbin engineering university; (Attention: 3 $\sim$ 6 key words, lower-case!)}



% 主要符号表

\NotationList{
  \begin{tabular}{ll}
    A & a matrix\\
    B & 登高\\
  \end{tabular}
}



% 生成封面

\makecover
\clearpage
