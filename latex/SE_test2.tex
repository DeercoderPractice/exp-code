\documentclass{article}
\usepackage{titlesec}
\usepackage[page]{totalcount}
\usepackage{indentfirst}
\author{University of Massachusetts, Lowell}
\title{Software Engineering(91.411) Exam1 Solution}

\begin{document}
\maketitle

\textbf{NOTE}: Each question is 10 points and the total score for this exam is 90 points.


\section{What is software engineering?}
An engineering discipline concerned with all aspects of software production from specification to system maintenance.

\section{What are the 3 general issues that affect many different types of software?}
Heterogeneity. Software may have to execute on several different types of system.

Business and social change, which drives requirements for software change.

Security and trust -- our software systems have to be secure against external and internal threats so that we can trust those systems.

\section{What software engineering fundamentals apply to all types of software systems?}
a. Systems should be developed using a managed and understood development process.

b. Dependability and performance are key system characteristics

c. Understanding and managing the software specification and requirements are important.

d. Effective use should be made of available resources.

\section{What are the fundamental activities that are common to all software processes?}
Software specification

Software design and implementation 

Software validation

Software evolution


\section{What are the three benefits of incremental development, compared to the waterfall model?}
(a) The cost of accommodating changes to customer requirements is reduced.

(b) It is easier to get customer feedback on development work that has been done.

(c) More rapid delivery and deployment of useful software to the customer is possible.

\section{What are the advantages of using incremental development and delivery?}
Early delivery of critical functionality to the customer

Early increments serve as prototypes to explore requirements

Lower risk of overall project failure

More extensive testing of critical customer functionality

\section{For what types of system are agile approaches to development particularly likely to be successful?}
Small and medium-sized software product development.

Custom software development in an organization where there is a clear commitment from customers to become involved in the development process.



\section{List 4 questions that should be asked when deciding whether or not to adopt an agile method of software development.}

\noindent Any 4 from those below. Others are also possible (see Ch 3) \newline

Is an incremental delivery strategy realistic?

What type of system is being developed?

What is the expected system lifetime?

How is the development team organized?

Is the system subject to external regulation?

How large is the system that is being developed?



\section{What are the barriers to introducing agile methods into large companies?}
Project managers may be reluctant to accept the risks of a new approach.

The established quality procedures in large companies may be incompatible with the informal approach to documentation in agile methods.

The existing teams may not have the high level of skills to make use of agile methods.

There may be cultural resistance if there is a long history of plan-driven development in the company.

\end{document}


