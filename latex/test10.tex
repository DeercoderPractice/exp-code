\documentclass{article}
\usepackage{amsmath}
\usepackage{amssymb}
\usepackage{graphicx, subfig}
\usepackage{caption}
\begin{document}
One image \ref{one-img}.
\begin{figure}[!htbp]
\centering
\includegraphics[width = .8\textwidth]{figure1.jpg}
\caption{example of one image} \label{one-img}
\end{figure}

Image together is shown in Figure \ref{img-together}.
First sub-image is shown as Figure \ref{sub1}.
In Figure \ref{sub2} the second sub-image is presented.
\begin{figure}[!htbp]
\centering
\subfloat[first sub-image]{
\includegraphics[width = .45\textwidth]{figure1.jpg}
\label{sub1}
}
\qquad
\subfloat[second sub-image]{
\includegraphics[width = .45\textwidth]{figure2.jpg}
\label{sub2}
}
\caption{combined image}\label{img-together}
\end{figure}

The result is shown in Equation \ref{abcde}:
\begin{equation}\label{abcde}
a+b+c+d+e=f
\end{equation}

\end{document}