\documentclass{article}
\usepackage{hyperref} %% use hper refere to url

\title{StatLearning: Statistical Learning \\ syllabus\footnote{Stanford Online Course}}
\author{Chang Liu\footnote{from \url{https://lagunita.stanford.edu/courses/HumanitiesandScience/StatLearning/Winter2015/info}}}

\begin{document}

\maketitle

\section{Introduction}

\indent Welcome to Statistical Learning!

Your learning adventure with Trevor, Rob and their team begins today.

The course follows closely the sequence of chapters in the course text ``An Introduction to Statistical Learning, with Applications in R" (James, Witten, Hastie, Tibshirani - Springer 2013). Remember, this textbook is available for free online at \url{http://www-bcf.usc.edu/~gareth/ISL/}.

The first week will be an overview of Statistical Learning, and will cover the first two chapters of the book. On each subsequent Saturday we will release the materials for a new chapter. Here is the schedule in detail:

Week 1: Introduction and Overview of Statistical Learning (Chapters 1-2, starts Jan 19)

Week 2: Linear Regression (Chapter 3, starts Jan 24)

Week 3: Classification (Chapter 4, starts Jan 31)

Week 4: Resampling Methods (Chapter 5, starts Feb 7)

Week 5: Linear Model Selection and Regularization (Chapter 6, starts Feb 14)

Week 6: Moving Beyond Linearity (Chapter 7, starts Feb 21)

Week 7: Tree-based Methods (Chapter 8, starts Feb 28)

Week 8: Support Vector Machines (Chapter 9, starts Mar 7)

Week 9: Unsupervised Learning (Chapter 10, starts Mar 14)

We hope that you will follow along with the class each week and share your comments and questions along the way. However, if you join the class late or are unavailable for a week, access to the material from previous weeks will remain open. The deadline for completing all the requirements to get your Statement of Accomplishment is April 3.
To pass the course you need to get 50\% or more correct answers on the quiz questions. If you score 90\% or higher, your statement will be ``with distinction".

To see the course materials, click on the Courseware tab on the upper left of the entry page. You'll see the sequence of sections in this course listed on the left. Click on a section to see its subsections, and click on a subsection to see its units, which contain videos, questions, etc. Within a given subsection, you can move from one unit to the next by clicking the next icons, which appear at the top and bottom of each page. The sequence of sections, subsections, and units is intended to be experienced in order.

Versions of the class videos are available for download. If you look under any section's video, you will see a ``Download video" button. If you click that, you can download and save the video on your device. These are not as high resolution as the class video, but if you also download the pdf slides for each chapter, you should be fine.
For further details on the course, click on the Courseware tab, open the ``Course Logistics" section, and click on both the ``Getting started" and "How to access the course textbook" subsections.

It's a pleasure to have you as part of the course. Enjoy the journey!
PS: Enrollment will remain open until March 21. If you are a fan of the course, please help us reach more students through social media. Please use Facebook, Twitter, or the social media of your choice to share:

1) Our Promo Video: \url{https://www.youtube.com/watch?v=St2-97n7atk}

2) Our Course Site: \url{https://statlearning.class.stanford.edu}


\section{Textbook Solutions}
Unofficial community-based, Github solutions: \url{http://blog.princehonest.com/stat-learning/}

\section{Last year's lecture materials}
Blog post with all of last year's lecture materials for those that want to skip ahead: 
\url{http://www.r-bloggers.com/in-depth-introduction-to-machine-learning-in-15-hours-of-expert-videos/}

\section{Textbook}

The course textbook is An Introduction to Statistical Learning, with Applications in R, by Gareth James, Daniela Witten, Trevor Hastie and Rob Tibshirani (Springer 2013).

You can download the pdf of the book for free from the authors' \href{http://www-bcf.usc.edu/~gareth/ISL/}{book website}. % href is used for generating url with text description.

Springer is also offering a 30\% discount on the purchase price of the printed book to students taking this online course. For this they should buy the book directly from Springer using the imbedded ISLR discount code. (Although it looks like only the ebook is discounted, if you click on the hardcover, you will see it is discounted too. You will get a price of \$55.99 including free shipping worldwide.)

Some course registrants may also be interested in the more advanced  book The Elements of Statistical Learning (second edition) by Trevor Hastie, Rob Tibshirani and Jerome Friedman (Springer 2009). The pdf of this book is also available from this book's website. Springer is also offering a 30\% discount on the purchase price of the printed  version of this book to students taking this online course. Again they should buy the book directly from Springer using this imbedded ESL discount code. (Although it looks like only the ebook is discounted, if you click on the hardcover, you will see it is discounted too)


\end{document}


